\documentclass[times, utf8, diplomski]{fer}
\usepackage{booktabs}

\begin{document}

% TODO: Navedite broj rada.
\thesisnumber{000}

% TODO: Navedite naslov rada.
\title{Proširenje LTZVisor monitora virtualnih strojeva za višejezgrene procesore}

% TODO: Navedite vaše ime i prezime.
\author{Magdalena Halusek}

\maketitle

% Ispis stranice s napomenom o umetanju izvornika rada. Uklonite naredbu \izvornik ako želite izbaciti tu stranicu.
\izvornik

% Dodavanje zahvale ili prazne stranice. Ako ne želite dodati zahvalu, naredbu ostavite radi prazne stranice.
\zahvala{}

\tableofcontents

\chapter{Uvod}
Neke od važnih odrednica ugradbenih računala su pouzdanost, predvidljivost i rad u stvarnom vremenu.
Povećanjem popularnosti IoT (engl. \textit{Internet of Things}) uređaja u industriji, počinje se isticati
još jedna dodatna karakteristika, a radi se o sigurnosti uređaja. Kako bi se uređaji u industriji osigurali,
ARM već dugi niz godina daje podršku Cortex-A procesorima \textit{TrustZone} sigurnosnim ekstenzijama.
Izlaskom M profila nove ARMv8 arhitekture, pojavila se i posebno razvijena \textit{TrustZone-M} sigurnosna
podrška Cortex-M procesora. Iako je \textit{TrustZone-M} sigurnosna ekstenzija razvijena ispočetka konkretno
za mikrokontrolere, krajnja funkcionalnost je slična onoj koju pruža \textit{TrustZone} ekstenzija Cortex-A
procesora.\\
U ovom radu će biti opisana \textit{TrustZone} sigurnosna ekstenzija ZedBoard platforme koja implementira
Cortex-A9 procesor s dvije jezgre. Kao primjer maksimalnog iskorištenja \textit{TrustZone} sigurnosne
ekstenzije, bit će prikazan LTZVisor monitor virtualnih strojeva (engl. \textit{hypervisor} ili engl.
\textit{Virtual Machine Monitor}). LTZVisor omogućava virtualizaciju u ugradbenim računalima, odnosno konkurentno
izvođenje dva operacijska sustava na jednoj jezgri procesora. Pojavom IoT uređaja u industriji, pojavila
se i sve veća potreba za virtualizacijom zbog potrebe za karakteristikama operacijskih sustava za rad u stvarnom
vremenu i operacijskih sustava opće namjene. Konkretno, karakteristike operacijskog sustava za rad u
stvarnom vremenu koje su potrebne su odziv u stvarnom vremenu, determinizam i pouzdanost. Operacijski
sustav opće namjene je često tražen zbog dobre mrežne podrške koju nudi. U nastavku će biti opisano kako
\textit{TrustZone} doprinosi izolaciji između navedenih operacijskih sustava. Pošto je LTZVisor monitor
virtualnih strojeva namijenjen jednoj jezgri procesora, u ovom radu će se opisati postupak prilagođenja
LTZVisora za obje jezgre ZedBoarda kako bi se maksimalno iskoristile mogućnosti platforme.

\chapter{Pokretanje i konfiguracija sustava Zynq-7000 platforme}

\chapter{Višejezgrene konfiguracije Zynq-7000 platforme}
Kao što je već napomenuto, ZedBoard implementira Cortex-A9 procesor s dvije jezgre te je za navedenu
platformu moguće razviti aplikacije s različitim konfiguracijama obzirom na broj jezgri koje su potrebne
u sustavu. Moguće je implementirati:
\begin{itemize}
  \item{Aplikacija namijenjena jednoj jezgri procesora (CPU0),}
  \item{Dvije aplikacije, svaka namijenjena jednoj jezgri i}
  \item{Aplikacija namijenjena objema jezgrama u sustavu.}
\end{itemize}
Važno je napomenuti da je jedna jezgra sustava (CPU0) primarna, a druga (CPU1) sekundarna. Prilikom
pokretanja sustava, na platformi je aktivna samo primarna jezgra, dok se sekundarna jezgra nalazi u
WFE (engl. \textit{Wait For Event}) stanju. Nakon što se sustav pokrenuo, primarna jezgra je zadužena
za buđenje sekundarne jezgre. Sekundarnu jezgru je moguće probuditi generiranjem događaja sustava
(engl. \textit{system event}), nakon čega sekundarna jezgra automatski skače na adresu koja je upisana
na lokaciji 0xFFFFFFF0. Jedan od načina na koji se može probuditi sekundarna jezgra je da aplikacija
koja se izvršava na CPU0 upiše adresu na kojoj se nalazi aplikacija za CPU1 na memorijsku lokaciju
0xFFFFFFF0 te izvrši SEV (engl. \textit{send event}) instrukciju koja rezultira buđenjem CPU1.
Komunikaciju između dvije jezgre je moguće ostvariti pomoću prekida između procesora
(\textit{inter-processor interrupt}), dijeljene memorije ili razmijene poruka.

\section{Asimetrično višejezgreno procesiranje}
Asimetrično višejezgreno procesiranje ili AMP (engl. \textit{Asymmetric Multiprocessing}) se odnosi na
konfiguraciju u kojoj svaki procesor (jezgra) izvršava svoju aplikaciju, odnosno gdje svaki procesor ima
svoju sliku operacijskog sustava. Dakle, kako bi bilo moguće razviti obje aplikacije, potrebno je koristiti
odvojene skripte za memorijsko povezivanje (\textit{linker} skripte) i skripte za pokretanje sustava. Oba
operacijska sustava dijele isti fizički memorijski prostor, odnosno ne postoji nikakva izolacija između dva
operacijska sustava. Karakteristike AMP konfiguracije su sljedeće:
\begin{itemize}
  \item{Većina uređaja mora biti posvećena određenom procesoru}
  \item{Upravitelj prekidima je dijeljen između procesora}
  \item{Samo jedan procesor je zadužen za inicijalizaciju upravitelja prekidima}
\end{itemize}


\section{Simetrično višejezgreno procesiranje}

\chapter{Zaključak}
Zaključak.

\bibliography{literatura}
\bibliographystyle{fer}

\begin{sazetak}
Sažetak na hrvatskom jeziku.

\kljucnerijeci{Ključne riječi, odvojene zarezima.}
\end{sazetak}

% TODO: Navedite naslov na engleskom jeziku.
\engtitle{Title}
\begin{abstract}
Abstract.

\keywords{Keywords.}
\end{abstract}

\end{document}
